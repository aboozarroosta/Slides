\documentclass{beamer}
\usepackage{default}
\usepackage{mathtools}
\usepackage{tabularx}
\usepackage[absolute,overlay]{textpos}

\usetheme{Berlin}
\title{Simulation of Pressure Field Generated by a Piston Transducer using Field II}
\date{July - September 2016}
\institute{BTU Cottbus-Senftenberg}
\author{Aboozar Roosta, Ren\'{e} Golinske}

\begin{document}
% introduction page
\begin{frame}
\titlepage\thispagestyle{empty}
\end{frame}

\setbeamertemplate{footline}[frame number]
% 
\begin{frame}{Task}
Simulation of the sound pressure generated by a circular piston transducer by means of Field II toolbox.

\begin{itemize}
 
\item Field II is a toolbox produced for MATLAB computing environment and is developed under C language and MATLAB.
\end{itemize}
\end{frame}

\begin{frame}{Theory}
The following illustration shows the flat piston transducer and variable declarations.
\begin{figure}
\includegraphics[scale=.5]{figures/pistonth}
\end{figure}
\centering{\textbf{Fig. 1}: The Flat piston. (From \textit{ref. [1]})}

\end{frame}

\begin{frame}{Theory}
\newtheorem{farfield}{Far-field}
\begin{farfield}
Assumption of a far enough distance can reduce calculation complexity. In the case of transducers this distance is defined as
$$N_D=\frac{D^2}{4\lambda}=\frac{R^2}{\lambda}$$
where D and R are diameter and radius of the transducer and $\lambda$ is the wavelength of the transducer excitation signal.
\end{farfield}
%{\tiny * An Elementary Introduction to Acoustics, Finn Jacobson, Technical University of Denmark, 2011} 
\end{frame}


\begin{frame}{Theory}
\newtheorem{pressurefield}{Sound Pressure}
\begin{pressurefield}
Under far-field assumption, the pressure of sound is calculated by:
$$\hat{p}(r,\theta)=\frac{j\omega\rho U}{r}\frac{aJ_1(ka\sin\theta)}{ka\sin\theta}e^{j(\omega t-kr)}\ \ \  \ \ \ \     *$$
where U is the amplitude of the excitation signal applied to the transducer.
\end{pressurefield}
{\tiny * An Elementary Introduction to Acoustics, Finn Jacobson, Technical University of Denmark, p.47, 2011} 
\end{frame}

\begin{frame}{Theory}
\newtheorem{directivity}{Directivity}
\begin{directivity}
The following formula is called directivity and measures the power density the transducer sends to a specific direction:
$$D(f,\theta)=\frac{2J_1(ka\sin\theta)}{ka\sin\theta}$$
Where it is usually drawn according to different values of ka which is essentially an indication of frequency times transducer radius:
$$ka=\frac{2\pi}{\lambda}a=\frac{2\pi}{c}fa$$
An illustration of directivity for the flat piston is presented in the next slide:
\end{directivity}
\end{frame}



\begin{frame}{Theory}
\begin{figure}
\includegraphics[scale=.35]{figures/directivityfigure}
\end{figure}
\centering{\textbf{Fig. 2}: Directivity of the flat piston as a function of ka. (From \textit{ref. [1]})}
\end{frame}

\begin{frame}{Theory}
\newtheorem{zaxis}{Z-axis pressure}
\begin{zaxis}
The pressure on the axis perpendicular to the center of the flat piston is computed as follows:
$$\hat{p}=2j\rho cUe^{j(wt-k[r+\Delta])}\sin (k\Delta)\ \ \  \ \ \ \     *$$
Where $\rho$ is the density, c is the speed of sound, U is the amplitude of the excitation signal and 
$$\Delta = \frac{\sqrt{r^2+a^2}-r}{2}$$
{\tiny * An Elementary Introduction to Acoustics, Finn Jacobson, Technical University of Denmark, p.48	, 2011} 
\end{zaxis}
\end{frame}
\begin{frame}{Theory - Field II}
Field II calculates the pressure field with following steps:
\begin{enumerate}
\item Divides the transducer to many small elements(rectangular, triangular or other polygons)
\item Since the far-field is much smaller for each element than the the whole transducer, it assumes far-field for every given point
\item Calculates the pressure by each element and adds them up to obtain the pressure signal
\end{enumerate}
\end{frame}

\begin{frame}{Theory - Field II}
This figure shows how Filed II divides a piston transducer to rectangular elements:
\begin{figure}
\includegraphics[scale=.4]{figures/piston_in_field}
\end{figure}
\end{frame}

\begin{frame}{Our setup}
In the following slides we will present results obtained by simulations.
Our setup characteristics:
\begin{itemize}
	\item Transducer: flat piston
	\item Diameter: 12.8mm
	\item Excitation wave: $f_{exc}=\sin(2\pi f_0)$
	\item Excitation frequency: $f_0=40000Hz$
	\item Environment: air
	\item Environment sound speed: $c=343\frac{m}{s}$ 
	\item Wavelength: $\lambda = 8.6mm$
	\item Far-field distance: $N_D=4.8mm$ 
\end{itemize}
\end{frame}

\begin{frame}{Code}
\begin{itemize}
\item Starts by defining the variables and initializing the Field II toolbox.
\item Calculates the pressure over a grid of points
\item Saves the data and plots the figure
\end{itemize}

\end{frame}

\begin{frame}{Code - Flowchart}
\begin{figure}[\centering]
\includegraphics[scale=.35]{flowchart/flowfull}
\end{figure}
\end{frame}

\begin{frame}{Code}
\begin{columns}[T] % align columns
\begin{column}[]{.05\textwidth}
\includegraphics[scale=.35]{flowchart/flow1}
\end{column}%
\hfill%
\begin{column}{.8\textwidth}
\small{First we initialize Field II and define transducer and environment variables:}
\newline
\newline
\includegraphics[scale=.39]{codepics/define_var_Initialise_field}
\end{column}%
\end{columns}
\end{frame}

\begin{frame}{Code}
	\begin{columns}[T] % align columns
		\begin{column}[]{.05\textwidth}
			\includegraphics[scale=.35]{flowchart/flow2}
		\end{column}%
		\hfill%
		\begin{column}{.8\textwidth}
			\small{Then we define the transducer and apply an excitation signal to it. We also define the grid and convert the transducer to triangular elements if necessary:}
			\newline
			\includegraphics[scale=.39]{codepics/define_transducer_grid}
		\end{column}%
	\end{columns}
\end{frame}

\begin{frame}{Code}
	\begin{columns}[T] % align columns
		\begin{column}[]{.05\textwidth}
			\includegraphics[scale=.35]{flowchart/flow3}
		\end{column}%
		\hfill%
		\begin{column}{.8\textwidth}
			\small{Then we calculate the response for every point in the grid, we use $streakm$ function to identify the steady state and then we save the mean absolute value of the response over the steady state:}
			\newline
			\includegraphics[scale=.39]{codepics/calculate_response}
		\end{column}%
	\end{columns}
\end{frame}

\begin{frame}{Code}
	\begin{columns}[T] % align columns
		\begin{column}[]{.05\textwidth}
			\includegraphics[scale=.35]{flowchart/flow4}
		\end{column}%
		\hfill%
		\begin{column}{.8\textwidth}
			\small{Finally we generate a file name, save the data and plot the results:}
			\newline
			\includegraphics[scale=.39]{codepics/save_data}
		\end{column}%
	\end{columns}
\end{frame}

\begin{frame}{Code - streakm Flowchart}
	\begin{figure}[\centering]
		\includegraphics[scale=.45]{flowchart/sflowall}
	\end{figure}
\end{frame}

\begin{frame}{Code - streakm}
	\begin{columns}[T] % align columns
		\begin{column}[]{.05\textwidth}
			\includegraphics[scale=.35]{flowchart/sflow1}
		\end{column}%
		\hfill%
		\begin{column}{.8\textwidth}
			\small{First we receive the pressure signal(and also moving average window length), we take a moving average of the signal and we define variables:}
			\newline
			\newline
			\includegraphics[scale=.39]{codepics/s_prepare_data}
		\end{column}%
	\end{columns}
\end{frame}

\begin{frame}{Code - streakm}
	\begin{columns}[T] % align columns
		\begin{column}[]{.05\textwidth}
			\includegraphics[scale=.35]{flowchart/sflow2}
		\end{column}%
		\hfill%
		\begin{column}{.8\textwidth}
			\small{Then we find parts of the signal where the moving average is stable to obtain the steady state:}
			\newline
			\newline
			\includegraphics[scale=.39]{codepics/s_find_streak}
		\end{column}%
	\end{columns}
\end{frame}

\begin{frame}{Code - streakm}
\begin{columns}[T] % align columns
	\begin{column}[]{.05\textwidth}
		\includegraphics[scale=.35]{flowchart/sflow3}
	\end{column}%
	\hfill%
	\begin{column}{.8\textwidth}
		\small{Finally we check for an unclosed range as error handling and we find the longest stable range and output its first and last sample index:}
		\newline
		\newline
		\includegraphics[scale=.39]{codepics/s_output_range}
	\end{column}%
\end{columns}
\end{frame}



\begin{frame}{Results}
We will perform our simulations with different Field II parameters and we will compare the results:
\begin{tabularx}{11.30cm}{m{2.5cm}| m{2cm}| m{2cm}|m{2cm}}
\ \ \ \ \ \ \ & \scriptsize{Element Type}  & \scriptsize{Element Size} & \scriptsize{Sampling Frequency}\\
\hline
\scriptsize{scenario R5-1M} & Rectangle & $\frac{wavelength}{5}$  & 1Mhz \\
\scriptsize{scenario R25-1M } & Rectangle & $\frac{wavelength}{25}$ & 1Mhz \\
\scriptsize{scenario R5-4M } & Rectangle & $\frac{wavelength}{5}$ & 4MHz\\
\scriptsize{scenario R25-4M } & Rectangle & $\frac{wavelength}{25}$ & 4MHz\\
\scriptsize{scenario T5-1M } & Triangle & $\frac{wavelength}{5}$ & 1MHz\\
\scriptsize{scenario T25-1M } & Triangle & $\frac{wavelength}{25}$ & 1MHz\\
\scriptsize{scenario T5-4M } & Triangle & $\frac{wavelength}{5}$ & 4Mhz \\
\scriptsize{scenario T25-4M } & Triangle & $\frac{wavelength}{25}$ & 4Mhz \\

\end{tabularx}


\end{frame}

\begin{frame}
Now we present the result of our simulation over these scenarios. First we demonstrate how our steady state detection works in different setups. We calculate the response in a plane that is perpendicular to the transducer.
\end{frame}

\begin{frame}{Moving Average Comparison - R5-1M}
 \begin{textblock*}{13cm}(0cm,1.7cm) % {block width} (coords)
		\includegraphics[width=13cm,height=7.5cm]{figures/rect6fs1MHz_size5thsix}
 \end{textblock*}   
\end{frame}

\begin{frame}{Moving Average Comparison - R25-1M}
 \begin{textblock*}{13cm}(0cm,1.7cm) % {block width} (coords)
		\includegraphics[width=13cm,height=7.5cm]{figures/rect6fs1MHz_size25thsix}
 \end{textblock*}   
\end{frame}




\begin{frame}{Moving Average Comparison - R5-4M}
 \begin{textblock*}{13cm}(0cm,1.7cm) % {block width} (coords)
		\includegraphics[width=13cm,height=7.5cm]{figures/rect6fs4MHz_size5thsix}
 \end{textblock*}   
\end{frame}

\begin{frame}{Moving Average Comparison - R25-4M}
 \begin{textblock*}{13cm}(0cm,1.7cm) % {block width} (coords)
		\includegraphics[width=13cm,height=7.5cm]{figures/rect6fs4MHz_size25thsix}
 \end{textblock*}   
\end{frame}


\begin{frame}{Moving Average Comparison - T5-1M}
 \begin{textblock*}{13cm}(0cm,1.7cm) % {block width} (coords)
		\includegraphics[width=13cm,height=7.5cm]{figures/tria6fs1MHz_size5thsix}
 \end{textblock*}   
\end{frame}


\begin{frame}{Moving Average Comparison - T25-1M}
 \begin{textblock*}{13cm}(0cm,1.7cm) % {block width} (coords)
		\includegraphics[width=13cm,height=7.5cm]{figures/tria6fs1MHz_size25thsix}
 \end{textblock*}   
\end{frame}

\begin{frame}{Moving Average Comparison - T5-4M}
 \begin{textblock*}{13cm}(0cm,1.7cm) % {block width} (coords)
		\includegraphics[width=13cm,height=7.5cm]{figures/tria6fs4MHz_size5thsix}
 \end{textblock*}   
\end{frame}

\begin{frame}{Moving Average Comparison - T25-4M}
 \begin{textblock*}{13cm}(0cm,1.7cm) % {block width} (coords)
		\includegraphics[width=13cm,height=7.5cm]{figures/tria6fs4MHz_size25thsix}
 \end{textblock*}   
\end{frame}

\begin{frame}{Pressure Field - Directivity}
Next we compare directivity and normalized pressure on the z-axis with theoretical results:
\end{frame}

\begin{frame}{Pressure Field - Directivity - R5-1M}
\begin{textblock*}{5cm}(.5cm,3.5cm) % {block width} (coords)
\includegraphics[width=5cm,height=3cm]{figures/rect6fs1MHz_size5th}
\newline \ \ \\ \ \ \ \ \ \ \ \ \ {\scriptsize Time(s) : 65}
\end{textblock*}
\begin{textblock*}{5cm}(6.5cm,2cm) % {block width} (coords)

	\includegraphics[width=5cm,height=3cm]{figures/rect6fs1MHz_size5z_axis}
	\\\ \ \ \ \ \ \ \ \ \ {\scriptsize Pressure on z-axis}
\end{textblock*}
\begin{textblock*}{5cm}(7cm,5.7cm) % {block width} (coords)
	\includegraphics[width=5cm,height=2cm]{figures/rect6fs1MHz_size5direc}
		\\\ \ \ \ \ \ \ \ \ \ {\scriptsize Directivity}
\end{textblock*}
\end{frame}

\begin{frame}{Pressure Field - Directivity - R25-1M}
\begin{textblock*}{5cm}(.5cm,3.5cm) % {block width} (coords)
\includegraphics[width=5cm,height=3cm]{figures/rect6fs1MHz_size25th}
\newline \ \ \\ \ \ \ \ \ \ \ \ \ {\scriptsize Time(s) : 128}
\end{textblock*}
\begin{textblock*}{5cm}(6.5cm,2cm) % {block width} (coords)

	\includegraphics[width=5cm,height=3cm]{figures/rect6fs1MHz_size25z_axis}
	\\\ \ \ \ \ \ \ \ \ \ {\scriptsize Pressure on z-axis}
\end{textblock*}
\begin{textblock*}{5cm}(7cm,5.7cm) % {block width} (coords)
	\includegraphics[width=5cm,height=2cm]{figures/rect6fs1MHz_size25direc}
		\\\ \ \ \ \ \ \ \ \ \ {\scriptsize Directivity}
\end{textblock*}
\end{frame}

\begin{frame}{Pressure Field - Directivity - R5-4M}
\begin{textblock*}{5cm}(.5cm,3.5cm) % {block width} (coords)
\includegraphics[width=5cm,height=3cm]{figures/rect6fs4MHz_size5th}
\newline \ \ \\ \ \ \ \ \ \ \ \ \ {\scriptsize Time(s) : 171}
\end{textblock*}
\begin{textblock*}{5cm}(6.5cm,2cm) % {block width} (coords)

	\includegraphics[width=5cm,height=3cm]{figures/rect6fs4MHz_size5z_axis}
	\\\ \ \ \ \ \ \ \ \ \ {\scriptsize Pressure on z-axis}
\end{textblock*}
\begin{textblock*}{5cm}(7cm,5.7cm) % {block width} (coords)
	\includegraphics[width=5cm,height=2cm]{figures/rect6fs4MHz_size5direc}
		\\\ \ \ \ \ \ \ \ \ \ {\scriptsize Directivity}
\end{textblock*}
\end{frame}

\begin{frame}{Pressure Field - Directivity - R25-4M}
\begin{textblock*}{5cm}(.5cm,3.5cm) % {block width} (coords)
\includegraphics[width=5cm,height=3cm]{figures/rect6fs4MHz_size25th}
\newline \ \ \\ \ \ \ \ \ \ \ \ \ {\scriptsize Time(s) : 239}
\end{textblock*}
\begin{textblock*}{5cm}(6.5cm,2cm) % {block width} (coords)

	\includegraphics[width=5cm,height=3cm]{figures/rect6fs4MHz_size25z_axis}
	\\\ \ \ \ \ \ \ \ \ \ {\scriptsize Pressure on z-axis}
\end{textblock*}
\begin{textblock*}{5cm}(7cm,5.7cm) % {block width} (coords)
	\includegraphics[width=5cm,height=2cm]{figures/rect6fs4MHz_size25direc}
		\\\ \ \ \ \ \ \ \ \ \ {\scriptsize Directivity}
\end{textblock*}
\end{frame}

\begin{frame}{Pressure Field - Directivity - T5-1M}
\begin{textblock*}{5cm}(.5cm,3.5cm) % {block width} (coords)
\includegraphics[width=5cm,height=3cm]{figures/tria6fs1MHz_size5th}
\newline \ \ \\ \ \ \ \ \ \ \ \ \ {\scriptsize Time(s) : 1971}
\end{textblock*}
\begin{textblock*}{5cm}(6.5cm,2cm) % {block width} (coords)

	\includegraphics[width=5cm,height=3cm]{figures/tria6fs1MHz_size5z_axis}
	\\\ \ \ \ \ \ \ \ \ \ {\scriptsize Pressure on z-axis}
\end{textblock*}
\begin{textblock*}{5cm}(7cm,5.7cm) % {block width} (coords)
	\includegraphics[width=5cm,height=2cm]{figures/tria6fs1MHz_size5direc}
		\\\ \ \ \ \ \ \ \ \ \ {\scriptsize Directivity}
\end{textblock*}
\end{frame}

\begin{frame}{Pressure Field - Directivity - T25-1M}
\begin{textblock*}{5cm}(.5cm,3.5cm) % {block width} (coords)
\includegraphics[width=5cm,height=3cm]{figures/tria6fs1MHz_size25th}
\newline \ \ \\ \ \ \ \ \ \ \ \ \ {\scriptsize Time(s) : 35856}
\end{textblock*}
\begin{textblock*}{5cm}(6.5cm,2cm) % {block width} (coords)

	\includegraphics[width=5cm,height=3cm]{figures/tria6fs1MHz_size25z_axis}
	\\\ \ \ \ \ \ \ \ \ \ {\scriptsize Pressure on z-axis}
\end{textblock*}
\begin{textblock*}{5cm}(7cm,5.7cm) % {block width} (coords)
	\includegraphics[width=5cm,height=2cm]{figures/tria6fs1MHz_size25direc}
		\\\ \ \ \ \ \ \ \ \ \ {\scriptsize Directivity}
\end{textblock*}
\end{frame}

\begin{frame}{Pressure Field - Directivity - T5-4M}
\begin{textblock*}{5cm}(.5cm,3.5cm) % {block width} (coords)
\includegraphics[width=5cm,height=3cm]{figures/tria6fs4MHz_size5th}
\newline \ \ \\ \ \ \ \ \ \ \ \ \ {\scriptsize Time(s) : 2361}
\end{textblock*}
\begin{textblock*}{5cm}(6.5cm,2cm) % {block width} (coords)

	\includegraphics[width=5cm,height=3cm]{figures/tria6fs4MHz_size5z_axis}
	\\\ \ \ \ \ \ \ \ \ \ {\scriptsize Pressure on z-axis}
\end{textblock*}
\begin{textblock*}{5cm}(7cm,5.7cm) % {block width} (coords)
	\includegraphics[width=5cm,height=2cm]{figures/tria6fs4MHz_size5direc}
		\\\ \ \ \ \ \ \ \ \ \ {\scriptsize Directivity}
\end{textblock*}
\end{frame}

\begin{frame}{Pressure Field - Directivity - T25-4M}
\begin{textblock*}{5cm}(.5cm,3.5cm) % {block width} (coords)
\includegraphics[width=5cm,height=3cm]{figures/tria6fs4MHz_size25th}
\newline \ \ \\ \ \ \ \ \ \ \ \ \ {\scriptsize Time(s) : 36126}
\end{textblock*}
\begin{textblock*}{5cm}(6.5cm,2cm) % {block width} (coords)

	\includegraphics[width=5cm,height=3cm]{figures/tria6fs1MHz_size25z_axis}
	\\\ \ \ \ \ \ \ \ \ \ {\scriptsize Pressure on z-axis}
\end{textblock*}
\begin{textblock*}{5cm}(7cm,5.7cm) % {block width} (coords)
	\includegraphics[width=5cm,height=2cm]{figures/tria6fs1MHz_size25direc}
		\\\ \ \ \ \ \ \ \ \ \ {\scriptsize Directivity}
\end{textblock*}
\end{frame}

\begin{frame}{Comparing all solutions}
Next we combine all previous figures for a side by side comparison.
\end{frame}

\begin{frame}{Comparing all solutions - Z-axis Pressure (scenarios 1-4)}
\begin{textblock*}{\textwidth}(1cm,2cm)
	\includegraphics[scale=.5]{figures/zaxis1_4}
\end{textblock*}
\end{frame}

\begin{frame}{Comparing all solutions - Z-axis Pressure (scenarios 5-8)}
\begin{textblock*}{\textwidth}(1cm,2cm)
	\includegraphics[scale=.5]{figures/zaxis5_8}
\end{textblock*}
\end{frame}

\begin{frame}{Comparing all solutions - Z-axis Pressure (scenarios 1-8)}
\begin{textblock*}{\textwidth}(1cm,2cm)
	\includegraphics[scale=.5]{figures/zaxis1_8}
\end{textblock*}
\end{frame}

\begin{frame}{Comparing all solutions - Directivity (scenarios 1-4)}
\begin{textblock*}{\textwidth}(1cm,2cm)
	\includegraphics[scale=.5]{figures/direc1_4}
\end{textblock*}
\end{frame}

\begin{frame}{Comparing all solutions - Directivity (scenarios 5-8)}
\begin{textblock*}{\textwidth}(1cm,2cm)
	\includegraphics[scale=.5]{figures/direc5_8}
\end{textblock*}
\end{frame}

\begin{frame}{Comparing all solutions - Directivity (scenarios 1-8)}
\begin{textblock*}{\textwidth}(1cm,2cm)
	\includegraphics[scale=.5]{figures/direc1_8}
\end{textblock*}
\end{frame}

\begin{frame}{Results - Numerical Comparison}
Now we present numerical criteria to evaluate the simulation accuracy.
\end{frame}

\begin{frame}{Table of comparison}

\begin{tabularx}{11.30cm}{m{1.5cm}| m{1.2cm}|m{1.5cm}|m{1.5cm}|m{1.5cm} |m{1.5cm}m{1.5cm}}


\ \ \ \ \ \ \ & \scriptsize{Time(s)}  & \scriptsize{Z-axis Max Error} & \scriptsize{Z-axis Mean Error}& \scriptsize{Directivity Max Error}& \scriptsize{Directivity Mean Error}\\
\hline
\scriptsize{R5-1M} & 65 & 6.53 & 3.88 & 12471 & 26.4\\
\scriptsize{R25-1M} & 128 & 1.38 & 1.93 & 994 & 6.2\\
\scriptsize{R5-4M} & 171 & 6.76 & 5.12 & 2496 & 26.6\\
\scriptsize{R25-4M} & 239 & 1.92 & 0.54 & 977 & 5.9\\
\scriptsize{T5-1M} & 1971 & 38.85 & 13.56 & 31998 & 357.6\\
\scriptsize{T25-1M} & 35856 & 13.98 & 27.90 & 50915 & 423\\
\scriptsize{T5-4M} & 2361 & 95.56 & 12.28 & 23477 & 195.9\\
\scriptsize{T25-4M} & 36126 & 26.16 & 17.48 & 27316 & 329.8\\


\end{tabularx}

\end{frame}

%References
\begin{frame}{References}
\begin{thebibliography}{99} % Beamer does not support BibTeX so references must be inserted manually as below
\bibitem[Rasmussen, 1996]{p1} 1. Rasmussen K. and Lydfelter (1996)
\newblock Acoustic Technology
\newblock \emph{Technical University of Denmark, Notes No. 2107}
\bibitem[Jacobsen, 2011]{p1} 2. Jacobsen Finn (2011)
\newblock An Elementary Introduction to Acoustics
\newblock \emph{Technical University of Denmark}
\end{thebibliography}
\end{frame}


\end{document}